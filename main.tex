\documentclass[10pt,a4paper]{article}
\usepackage[utf8]{inputenc}
\usepackage[czech]{babel} % použijte "slovak" namísto "czech", pokud píšete slovensky
\usepackage[T1]{fontenc}
\usepackage{graphicx}
\usepackage{lmodern}
\usepackage[top = 2cm, bottom = 2cm, left = 2cm, right = 2cm]{geometry}

    
%For title page
\usepackage{titlesec}
\usepackage{setspace} %Rámeček nahoře
\usepackage{framed} %Rámeček nahoře
\usepackage{array} %Tabulka dole

\usepackage{caption}
\usepackage{subcaption}
\usepackage{url}
\usepackage{booktabs}
\usepackage{newfloat}
\usepackage{multirow}
\usepackage{dirtytalk}
\usepackage{xspace}
\usepackage{upgreek}
\usepackage{placeins}
\usepackage{indentfirst}
\usepackage{longtable}
\usepackage{siunitx}
\usepackage{hyperref}
\usepackage{amsmath}
\usepackage{multirow}

\DeclareFloatingEnvironment{graph}
\addto\captionsczech{%
  \renewcommand{\graphname}{Graf}%
  \renewcommand{\figurename}{Obr.}%
}
\addto\captionsslovak{%
  \renewcommand{\graphname}{Graf}%
  \renewcommand{\figurename}{Obr.}%
}


\bibliographystyle{ieeetr}


%-------------------- INFORMACE NA PREDNI STRANE PROTOKOLU ------------------------
%CISLO PRAKTIKA
\newcommand\cisloPraktika{x} % I, II, III, nebo IV
%JMENO
\newcommand\jmeno{xxx xxx}
%MAIL
\newcommand\mail{xxx@xxx.xxx}

%NÁZEV ÚLOHY
\newcommand\nazevUlohy{xxx}
%ČÍSLO ÚLOHY
\newcommand\cisloUlohy{xxx}
%DATUM MERENI
\newcommand\datumMereni{dd.\,m.\,yyyy} %format: 30.\,9.\,2021 
%----------------------------------------------------------------------------------
 

\begin{document}
\thispagestyle{empty}
\newgeometry{top = 2.5cm, bottom = 0cm, left = 2.5cm, right = 3cm}

{%T tomto je uzavřena celá titulka
%Tloušťka rámečku
\setlength{\fboxrule}{1.5pt}

\noindent
\framebox{
\begin{minipage}{\textwidth}
\setlength{\parindent}{17.62482 pt}
\phantom{d}

\begin{minipage}{0.6\textwidth}
{
\Large Kabinet výuky obecné fyziky, UK MFF\\
}
\vspace*{0.2cm}

{
\bfseries
\huge Fyzikální praktikum \cisloPraktika%ČÍSLO
}
\end{minipage}
\begin{minipage}{0.4\textwidth}
\begin{center}
\includegraphics[width=4.5cm]{ZFP.jpg}
\end{center}
\end{minipage}\\\\

%\vspace*{0.5cm}

{
\setstretch{1.5}
\Large
\noindent
Úloha č. \cisloUlohy%Číslo

\noindent
Název úlohy: \nazevUlohy%Název

\noindent
Jméno: \jmeno%Jméno

\noindent
Email: \mail%Mail

\noindent
Datum měření: \datumMereni%Datum

\phantom{d}
}
\end{minipage}
}
%Konec horního rámečku

{
\phantom{d}

\Large
Připomínky opravujícího:\\
\vspace*{6.75cm}
}

\newcommand{\linka}{\noalign{\hrule height 1pt}}
\newcommand{\linkadva}{\noalign{\hrule height 1.5pt}}
\setlength\extrarowheight{9.5pt}
\Large
\noindent
\begin{tabular}{!{\vrule width 1.5pt} l !{\vrule width 1pt} c !{\vrule width 1pt} c !{\vrule width 1.5pt}}
\linkadva
   & Možný počet bodů & Udělený počet bodů \\\linkadva
  Teoretická část & 0--2 &  \\\linka
  Výsledky a zpracování měření & 0--9 &  \\\linka
  Diskuse výsledků & 0--4 &  \\\linka
  Závěr & 0--1 &  \\\linka
  Použitá literatura & 0--1 &  \\\linkadva
  \hspace*{\fill} \textbf{Celkem} \hspace*{\fill}& max. 17 &  \\
\linkadva
\end{tabular}
\phantom{d}

Posuzoval: \hspace*{\fill}dne:~~~~~~~~~~~~~~~~~

}%Konec uzavření titulky
\newpage
\newgeometry{top = 2cm, bottom = 2cm, left = 2cm, right = 2cm}
\setcounter{page}{1}



% Následující dva řádky před přípravou protokolu zakomentujte, nebo vymažte!!!
\section{Ukázky}
Ukázky, jak se co dělá.
Literatura se cituje takto~\cite{broz}.
Kód \texttt{broz} pochází ze souboru Praktika.bib, ve kterém jsou bibtex údaje o mnoha zdrojích, které by se Vám mohly hodit.
Soubor si prohlédněte, vyberte si své zdroje a pak už jen citujte.
Soubor také můžete doplnit o další položky!

Matematické rovnice lze psát více způsoby.
Přímo v textu se píší takto: $f(x) = ax + b$.
Rovnice, které chcete patřičně zdůraznit (to bude nejspíše většina), můžete psát např. takto:
\begin{equation}
  \label{eq:lin_func}
  f(x) = ax + b
\end{equation}
Na rovnici se pak odkazujete takto: rovnice~\eqref{eq:lin_func}.

Obrázky vložíte takto.
\begin{figure}
  \centering
  \includegraphics[width=0.7\textwidth]{figures/example_figure.pdf}
  \caption{Příklad změřené závislosti proudu na napětí nafitované lineární funkcí.
    Černé čtverečky zobrazují výsledky měření.
    Červená čára ukazuje nafitovanou lineární závislost.
  }
  \label{fig:i_vs_u}
\end{figure}
Na obrázek se pak odkážete takto: obr.~\ref{fig:i_vs_u}.

Tabulka se vloží takto.
\begin{table}
\centering
\begin{tabular}{rcl}
\hline
U [V] & I [A] & R[$\Omega$] \\
\hline
10.0 & 0.1 & 100.0 \\
29.0 & 0.26 & 111.5 \\
51.0 & 0.54 & 94.4 \\
70.0 & 0.71 & 98.6 \\
90.0 & 0.97 & 92.8 \\
111.0 & 1.11 & 100.0 \\
130.0 & 1.37 & 94.9 \\
149.0 & 1.4 & 106.4 \\
171.0 & 1.54 & 111.0 \\
189.0 & 1.86 & 101.6 \\
\hline
\end{tabular}
\caption{Změřené hodnoty napětí $U$ a proudu $I$.
  Hodnoty odporu $R$ jsou spočítané z Ohmova zákona.}
\label{tab:ohm}
\end{table}
Na tabulku se odkazuje stejným způsobem jako na obrázek, tj. tab.~\ref{tab:ohm}.
\begin{table}
\centering
\begin{tabular}{ccc}
\hline
U [V] & I [A] & R[$\Omega$] \\
\hline
51.0 & 0.54 & 94.4 \\
70.0 & 0.71 & 98.6 \\
\multicolumn{2}{c}{-} & 111.5 \\
90.0 & \multicolumn{2}{c}{\multirow{2}{*}{-}} \\
111.0 &  \\
\multirow{3}{*}{-} & 1.37 & 94.9 \\
 & 1.4 & 106.4 \\
 & 1.54 & 111.0 \\
189.0 & 1.86 & 101.6 \\
\hline
\end{tabular}
\caption{V tabulce jsou sloučené sloupce a řádky pomocí příkazu \texttt{multicolumn} a \texttt{multirow}.
  Fyzikákně nedává tabulka smysl, je to jen příklad.}
\label{tab:ohm2}
\end{table}
Tab.~\ref{tab:ohm2} je příklad tabulky se sloučenými sloupci a řádky pomocí příkazu \texttt{multicolumn} a \texttt{multirow}.

\subsection{Matematické vzorce podrobněji}
\label{sec:mvp}

Používají se prostředí \texttt{equation} pro jednořádkové rovnice a \texttt{align} pro víceřádkové rovnice.
Případně můžete použít i \texttt{equation*} a \texttt{align*}, pokud nechcete, aby byly rovnice číslovány.
Matematické vzorce se dají psát i víceřádkově.
\begin{align}
  f(x) &= ax + b \\
  g(x, y) &= ax + by^{2} + \sqrt{xy} \\
  h(x) &= \left(\frac{x}{1 + x^2}\right)^3 \\
  N(x; \mu, \sigma) &= \frac{1}{\sqrt{2\pi\sigma^2}} e^{-\frac{(x-\mu)^2}{2\sigma^2}} \\
  \frac{\partial f}{\partial x} &= a \\
\end{align}
Řecká písmena se píší takto:
\begin{equation*}
  \alpha \beta \gamma \delta \epsilon \zeta \eta \theta \iota \kappa \lambda \mu \nu \xi \pi \rho \sigma \tau \upsilon \phi \chi \psi \omega
\end{equation*}
a velká řecká písmena takto (všimněte si, že některá se píší stejně jako latinská):
\begin{equation}
  A B \Gamma \Delta E Z H \Theta I K \Lambda M N \Xi \Pi P \Sigma T \Upsilon \Phi X \Psi \Omega
\end{equation}
Horní a dolní indexy se píší následovně:
\begin{equation}
  x^2 \quad x_2 \quad x^{2+3} \quad x_{2+3} \quad x^{2+3}_{2+3}
\end{equation}
Vektorové operace se píší takto:
\begin{equation}
  \nabla \cdot \vec{E} \quad \nabla \times \vec{B} \quad \nabla \cdot \nabla \times \vec{A} \quad \nabla \times \nabla \cdot \vec{A}
\end{equation}
Integrály se píší takto:
\begin{equation}
  \int f(x) \, \mathrm{d}x \quad \int_a^b f(x) \, \mathrm{d}x \quad \iint f(x, y) \, \mathrm{d}x \, \mathrm{d}y \quad \iiint f(x, y, z) \, \mathrm{d}x \, \mathrm{d}y \, \mathrm{d}z
\end{equation}
Sumy se píší takto:
\begin{equation}
  e^{x} = \sum_{n=0}^{\infty} \frac{x^n}{n!}
\end{equation}


\section{Pracovní úloha}
\label{section:uloha}
\input{sections/pracovni_uloha.tex}

\section{Teorie}
\label{section:teorie}
\input{sections/teorie.tex}
\FloatBarrier

\section{Výsledky a zpracování měření}
\label{section:zpracovani}
\input{sections/vysledky_a_zpracovani_mereni.tex}
\FloatBarrier


\section{Diskuze}
\label{section:diskuze}
\input{sections/diskuze.tex}
\FloatBarrier

\section{Závěr}
\label{section:zaver}
\input{sections/zaver.tex}
\FloatBarrier
\bibliography{Praktika.bib}

\newpage
\section{Přílohy}
% Napr. tabulky:
%\input{tables/tabulka_do_prilohy.tex}

\end{document}

