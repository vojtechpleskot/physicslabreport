Ukázky, jak se co dělá.
Literatura se cituje takto~\cite{broz}.
Kód \texttt{broz} pochází ze souboru Praktika.bib, ve kterém jsou bibtex údaje o mnoha zdrojích, které by se Vám mohly hodit.
Soubor si prohlédněte, vyberte si své zdroje a pak už jen citujte.
Soubor také můžete doplnit o další položky!

Matematické rovnice lze psát více způsoby.
Přímo v textu se píší takto: $f(x) = ax + b$.
Rovnice, které chcete patřičně zdůraznit (to bude nejspíše většina), můžete psát např. takto:
\begin{equation}
  \label{eq:lin_func}
  f(x) = ax + b
\end{equation}
Na rovnici se pak odkazujete takto: rovnice~\eqref{eq:lin_func}.

Obrázky vložíte takto.
\begin{figure}
  \centering
  \includegraphics[width=0.7\textwidth]{figures/example_figure.pdf}
  \caption{Příklad změřené závislosti proudu na napětí nafitované lineární funkcí.
    Černé čtverečky zobrazují výsledky měření.
    Červená čára ukazuje nafitovanou lineární závislost.
  }
  \label{fig:i_vs_u}
\end{figure}
Na obrázek se pak odkážete takto: obr.~\ref{fig:i_vs_u}.

Tabulka se vloží takto.
\begin{table}
\centering
\begin{tabular}{rcl}
\hline
U [V] & I [A] & R[$\Omega$] \\
\hline
10.0 & 0.1 & 100.0 \\
29.0 & 0.26 & 111.5 \\
51.0 & 0.54 & 94.4 \\
70.0 & 0.71 & 98.6 \\
90.0 & 0.97 & 92.8 \\
111.0 & 1.11 & 100.0 \\
130.0 & 1.37 & 94.9 \\
149.0 & 1.4 & 106.4 \\
171.0 & 1.54 & 111.0 \\
189.0 & 1.86 & 101.6 \\
\hline
\end{tabular}
\caption{Změřené hodnoty napětí $U$ a proudu $I$.
  Hodnoty odporu $R$ jsou spočítané z Ohmova zákona.}
\label{tab:ohm}
\end{table}
Na tabulku se odkazuje stejným způsobem jako na obrázek, tj. tab.~\ref{tab:ohm}.
\begin{table}
\centering
\begin{tabular}{ccc}
\hline
U [V] & I [A] & R[$\Omega$] \\
\hline
51.0 & 0.54 & 94.4 \\
70.0 & 0.71 & 98.6 \\
\multicolumn{2}{c}{-} & 111.5 \\
90.0 & \multicolumn{2}{c}{\multirow{2}{*}{-}} \\
111.0 &  \\
\multirow{3}{*}{-} & 1.37 & 94.9 \\
 & 1.4 & 106.4 \\
 & 1.54 & 111.0 \\
189.0 & 1.86 & 101.6 \\
\hline
\end{tabular}
\caption{V tabulce jsou sloučené sloupce a řádky pomocí příkazu \texttt{multicolumn} a \texttt{multirow}.
  Fyzikákně nedává tabulka smysl, je to jen příklad.}
\label{tab:ohm2}
\end{table}
Tab.~\ref{tab:ohm2} je příklad tabulky se sloučenými sloupci a řádky pomocí příkazu \texttt{multicolumn} a \texttt{multirow}.

\subsection{Matematické vzorce podrobněji}
\label{sec:mvp}

Používají se prostředí \texttt{equation} pro jednořádkové rovnice a \texttt{align} pro víceřádkové rovnice.
Případně můžete použít i \texttt{equation*} a \texttt{align*}, pokud nechcete, aby byly rovnice číslovány.
Matematické vzorce se dají psát i víceřádkově.
\begin{align}
  f(x) &= ax + b \\
  g(x, y) &= ax + by^{2} + \sqrt{xy} \\
  h(x) &= \left(\frac{x}{1 + x^2}\right)^3 \\
  N(x; \mu, \sigma) &= \frac{1}{\sqrt{2\pi\sigma^2}} e^{-\frac{(x-\mu)^2}{2\sigma^2}} \\
  \frac{\partial f}{\partial x} &= a \\
\end{align}
Řecká písmena se píší takto:
\begin{equation*}
  \alpha \beta \gamma \delta \epsilon \zeta \eta \theta \iota \kappa \lambda \mu \nu \xi \pi \rho \sigma \tau \upsilon \phi \chi \psi \omega
\end{equation*}
a velká řecká písmena takto (všimněte si, že některá se píší stejně jako latinská):
\begin{equation}
  A B \Gamma \Delta E Z H \Theta I K \Lambda M N \Xi \Pi P \Sigma T \Upsilon \Phi X \Psi \Omega
\end{equation}
Horní a dolní indexy se píší následovně:
\begin{equation}
  x^2 \quad x_2 \quad x^{2+3} \quad x_{2+3} \quad x^{2+3}_{2+3}
\end{equation}
Vektorové operace se píší takto:
\begin{equation}
  \nabla \cdot \vec{E} \quad \nabla \times \vec{B} \quad \nabla \cdot \nabla \times \vec{A} \quad \nabla \times \nabla \cdot \vec{A}
\end{equation}
Integrály se píší takto:
\begin{equation}
  \int f(x) \, \mathrm{d}x \quad \int_a^b f(x) \, \mathrm{d}x \quad \iint f(x, y) \, \mathrm{d}x \, \mathrm{d}y \quad \iiint f(x, y, z) \, \mathrm{d}x \, \mathrm{d}y \, \mathrm{d}z
\end{equation}
Sumy se píší takto:
\begin{equation}
  e^{x} = \sum_{n=0}^{\infty} \frac{x^n}{n!}
\end{equation}
